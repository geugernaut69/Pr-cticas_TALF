\documentclass{article}
\usepackage[utf8]{inputenc}
\usepackage{ragged2e}
\usepackage{graphicx}
\graphicspath{ {/home/user/Imágenes/} }

\title{Práctica1}
\author{Enrique Narbona Luque}
\date{October 2023}

\begin{document}

\maketitle

\LARGE
\titl\underline{Actividades}

\vspace{5mm}

\RaggedRight
\normalsize

\textbf{1. Find the power set R3 of R = {(1, 1), (1, 2), (2, 3), (3, 4)}. Check your answer with the script powerrelation.m and write a LATEX document with the solution step by step.}

\vspace{5mm}

\normalsize
\RaggedRight

Para llegar hasta este resultado, primero, se ha calculado R² y luego R³:

\vspace{2mm}

\begin{equation}
    R^{2} = R \circ R 
\end{equation}

\vspace{1mm}

\begin{equation}
     R^{3} = R^{2} \circ R
\end{equation}

\vspace{2mm}

\RaggedRight
Para calcular R² aplicamos la propiedad transitiva (a,b) \in R \wedge (b,c) \in R \rightarrow (a,c) \in R, obteniendo lo siguiente:

\vspace{3mm}

\begin{equation}
    R = {(1,1),(1,2),(1,3),(1,4)}
\end{equation}

\vspace{2mm}

Esto se debe a lo siguiente:

\vspace{3mm}

\underline{1}R1 \rightarrow 1R\underline{1} 1R\underline{2}

\vspace{1mm}

\underline{1}R2 \rightarrow 2R\underline{3}

\vspace{1mm}

\underline{2}R3 \rightarrow 3R\underline{4}

\vspace{4mm}

Resultado: R^2 = R \circ R = \{(1,1)(1,2)(1,3)(2,4)\}

\vspace{8mm}

\newpage

Ahora, calculamos R³:

\vspace{4mm}

\underline{1}R2 \rightarrow \underline{1}R\underline{2}

\vspace{1mm}

\underline{1}R2 \rightarrow 2R\underline{3}

\vspace{1mm}

\underline{1}R3 \rightarrow 3R\underline{4}

\vspace{4mm}

Resultado: R³ = R² \circ R = \{(1,1),(1,2),(1,3),(1,4)\}

\vspace{6mm}

A continuación para comprobar el resultado con octave, ejecutamos en la carpeta donde se encuentra el archivo powerrelation.m, el script help seguido del nombre de dicho archivo, el cual nos devolverá lo siguiente:

\vspace{4mm}

Power n of a relation R, or its transitive closure (if n undefined)

\vspace{4mm}

Examples

\vspace{3mm}

   {(a, b), (c, c), (b, a)}³
   
\vspace{1mm}
   
powerrelation({['a', 'b'], ['c', 'c'], ['b', 'a']}, 3)
   
\vspace{1mm}

ans = \{[1,1] = ab, [1,2] = ba, [1,3] = cc]\}
    
\vspace{5mm}

(a, b), (c, c), (b, a)∞
   
\vspace{1mm}

powerrelation(\{['a', 'b'], ['c', 'c'], ['b', 'a']\{)
   
\vspace{1mm} 

ans = \{[1,1] = aa, [1,2] = ab, [1,3] = ba, [1,4] = bb, [1,5] = cc\}
   
\vspace{5mm}

Ahora ya sabemos que debemos utilizar el siguiente comando en octave: powerrelation(conjunto, valor del exponente). Por tanto debemos escribir powerrelation({['1', '1'],['1', '2'],['2', '3'],['3', '4']},3)

\vspace{5mm}

Tras ejecutar el comando anterior, recibiremos el siguiente resultado:

\vspace{1mm}

ans = \{[1,1] = 11, [1,2] = 12, [1,3] = 13, [1,4] = 14\}

\vspace{3mm}

Por tanto, podemos ver que nuestra respuesta está bien.

\newpage

\normalsize
\RaggedRight

\textbf{2. Within the folder “files”, find a TEX file in whose content appears the string \usepackage{amsthm, amsmath}. Note: use grep and escape the special characters with \. Complete the proof and answer the question.}

\vspace{5mm}

{Lo primero que vamos a hacer es buscar los archivos mediante el comando grep "\usepackage{amsthm, amsmath}" ./*}

\vspace{4mm}

Como podemos observar este script nos ha devuelto el archivo, el cual contiene escrito lo que hemos especificado en el comando.

\vspace{4mm}

Si abrimos dicho archivo, mainP.tex, podremos ver que, efectivamente, este contiene la cadena \backslash usepackage\{amsthm, amsmath\}.

\vspace{4mm}

Además, podemos ver que al final de este archivo, se nos pide completar el último apartado:

\vspace{4mm}

\textbf{Consideremos $L=\{w\in \{a,b\}^* : w \textnormal{ no termina en } ab\}$. Una expresión regular que genera L es:}

\begin{equation}
    L(a^{\ast} + b^{\ast}a) =\{a,aa,aba,abba,ba,bba...\}
\end{equation}




\end{document}
