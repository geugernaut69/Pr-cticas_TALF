\documentclass{article}
\usepackage[utf8]{inputenc}
\usepackage{ragged2e}
\usepackage{graphicx}
\graphicspath{ {/home/user/Imágenes/} }

\title{Práctica1}
\author{Enrique Narbona Luque}
\date{October 2022}

\begin{document}

\maketitle

\LARGE
\titl\underline{Actividades}

\normalsize
\RaggedRight

\textbf{2. Within the folder “files”, find a TEX file in whose content appears the string \usepackage{amsthm, amsmath}. Note: use grep and escape the special characters with \. Complete the proof and answer the question.}

\vspace{5mm}

{Lo primero que vamos a hacer es buscar los archivos mediante el comando grep "\usepackage{amsthm, amsmath}" ./*}

\vspace{4mm}

Como podemos observar este script nos ha devuelto el archivo, el cual contiene escrito lo que hemos especificado en el comando.

\vspace{4mm}

Si abrimos dicho archivo, mainP.tex, podremos ver que, efectivamente, este contiene la cadena \backslash usepackage\{amsthm, amsmath\}.

\vspace{4mm}

Además, podemos ver que al final de este archivo, se nos pide completar el último apartado:

\vspace{4mm}

\textbf{Consideremos $L=\{w\in \{a,b\}^* : w \textnormal{ no termina en } ab\}$. Una expresión regular que genera L es:}

\begin{equation}
    L(a^{\ast} + b^{\ast}a) =\{a,aa,aba,abba,ba,bba...\}
\end{equation}

\end{document}